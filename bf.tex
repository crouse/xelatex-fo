\documentclass[UTF8]{ctexbook}
\usepackage{xcolor}
\usepackage{fontspec}
\usepackage{hyperref}
\usepackage{xeCJK}
\usepackage{xpinyin}
\setCJKmainfont{KaiTi}

\begin{document}

\title{大乘百法明门论}
\author{世亲菩萨造 玄奘法师译}
\maketitle
\LARGE

\begin{pinyinscope}
如世尊言:“ 一切法无我.”
何等一切法? 云何为无我?
一切法者, \xpinyin{略}{lue4}有五种: 一者心法, 二者心所有法, 三者色法, 四者心不相应行法, 五者无为法。 一切最胜故, 与此相应故, 二所现影故, 三位差别故, 四所显示故, 如是次第。

第一心法, \xpinyin{略}{lue4}有八种: 一眼识, 二耳识,三鼻识, 四舌识, 五身识, 六意识, 七末那识, 八阿赖耶识。

第二心所有法, \xpinyin{略}{lue4}有五十一种, 分为六位: 一遍行有五, 二别境有五, 三善有十一, 四烦恼有六, 五随烦恼有二十, 六不定有四。
一遍行五者: 一作意, 二触, 三受, 四想, 五思。
二别境五者: 一欲, 二胜解, 三念, 四定, 五慧。
三善十一者: 一信, 二精进, 三惭, 四愧, 五无贪, 六无嗔, 七无痴, 八轻安, 九不放逸, 十行舍, 十一不害。
四烦恼六者: 一贪, 二嗔, 三痴, 四慢, 五疑, 六不正见。
五随烦恼二十者: 一忿, 二恨, 三恼, 四覆, 五诳, 六谄, 七骄, 八害, 九嫉, 十悭, 十一无惭, 十二无愧, 十三不信, 十四懈怠, 十五放逸, 十六昏沉, 十七掉举, 十八散乱, 十九失念, 二十不正知。
六不定四者: 一悔, 二眠, 三寻, 四伺。

第三色法, \xpinyin{略}{lue4}有十一种: 一眼, 二耳, 三鼻, 四舌, 五身, 六色, 七声, 八香, 九味, 十触, 十一法处所摄色。

第四心不相应行法, \xpinyin{略}{lue4}有二十四种: 一得, 二命根, 三众同分, 四异生性, 五无想定, 六灭尽定, 七无想报, 八名身, 九句身, 十文身, 十一生, 十二住, 十三老, 十四无常, 十五流转, 十六定异, 十七相应, 十八势速, 十九次第, 二十时, 二十一方, 二十二数, 二十三和合性, 二十四不和合性。
 
 第五无为法者, \xpinyin{略}{lue4}有六种: 一虚空无为, 二择灭无为, 三非择灭无为, 四不动无为, 五想受灭无为, 六真如无为。
言无我者, \xpinyin{略}{lue4}有二种: 一补特伽罗无我, 二法无我。
\end{pinyinscope}

\end{document}
